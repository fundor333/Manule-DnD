
\chapter{Regole di base}
\DndDropCapLine{Q}ui indico le regole base per la campagna.
Se non vengono accettate tutte dai giocatori non si gioca.
Qui seguiamo la regola del "Patti chiari e amicizia lunga"

\section{Creazione del personaggio}

\subsection{Manuali consentiti}
I manuali consentiti sono i seguenti e sono divisi per Master e Giocatori

\begin{DndTable}{cXXX}
\textbf{Manuale}&  \textbf{PG} & \textbf{DM} \\
Manuale del Giocatore  & Si & Si \\
Manuale dei Mostri  & No & Si \\
Manuale del DM  & Si & Si \\
Calderone Omnicomprensivo di Tasha  & Si & Si \\
Mordenkainen presenta: Mostri del Multiverso  & Si & Si\\
Guida Omnicomprensiva di Xanathar  & Si & Si\\
Guildmasters' Guide to Ravnica  & No & Si\\
Espansione per la campagna & Si & Si\\
\end{DndTable}

In ogni momento questo è anche l'ordine per decidere quale manuale ha ragione: dall' alto in basso, più è in basso nella lista più ha ragione.

\begin{DndComment}{Definire chi ha ragione}
	Per esempio se il Manuale del Giocatore va in conflitto con Tasha, visto che Tasha è più in basso nella tabella, ha ragione Tasha.

	In ogni caso il DM ha l'ultima parola e quello che viene deciso verrà poi messo in questo libretto in modo che non ci sia più dubbio su quanto già deciso
\end{DndComment}

\subsection{Cosa posso usare per creare il PG?}
Quelli indicati prima quando è segnato che il giocatore lo può usare per il PG.
Lo faccio sia per limitare i problemi di sovrapposizione di cose visto che ci sono parti ripetute.
Segno anche che manuali uso come DM perchè mi pare divertente darvi delle fonti e vedervi spaventati per cosa vi aspettate.

\subsection{Dadi caratteristica per i personaggi}
Sono un vecchio giocatore e quindi ho visto costruire tanti personaggi ma non sono mai stato soddisfatto dei tiri a meno di fare in questo modo:

Si tirano 5 D6 per 8 volte. A ogni gruppo di 5 tolgo il risultato più alto e il più basso e poi di questi 8 risultati si toglie il più alto e il più basso.

\begin{DndComment}{Non ho capito i dadi per le caratteristiche}
I tiri vanno fatti davanti al master col master che controlla tutto quindi se non capite come fare tranquilli che facciamo assieme
\end{DndComment}

\subsection{Io pensavo di...}
I desiderata su come evolvere i personaggi sono sia una cosa positiva ma anche negativa.
Io non sono contrario al fatto che il giocatore abbia già una idea di come sviluppare il PG ma desidero che sia una crescita naturale e non qualcosa di sforzato. Quindi se c'è il desiderio di evolvere il personaggio in un certo modo (tipo multiclasse) si concorda la cosa in modo che il DM e il Giocatore rendano l'evoluzione del personaggio naturale e non spinta

\subsection{Morte di un PG}
I PG moriranno. Non sono contrario a farli morire e non sono nemmeno contrario a farli morire in modo definitivo. L'importante è che sia divertente per tutti e che funzioni per la storia. Può anche succedere che il PG muoia e inizi una mini quest per salvarlo/rianimarlo o simili.
